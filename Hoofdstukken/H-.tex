\chapter{Tijdsplanning en communicatie}
Om er voor te zorgen dat er consistent gewerkt wordt aan het productverslag wordt er zo veel mogelijk gehouden aan de planning en normen die de school stelt en voorschrijft. Er wordt middels een eenvoudig Excel-bestand dagelijks bijgehouden hoe het aantal uren zich verhoudt ten opzichte van de opgelegde norm vanuit school, zodat er elke dag inzicht is in wat ruwweg de stand is. Ook is er een geautomatiseerde grafiek ingevoegd zodat snel kan worden gezien waar de meeste uren aan worden besteed. Er wordt een veiligheidsmarge van +10\% bijgeteld boven op deze urennorm. 

\begin{figure}[h]
    \centering
    \includegraphics[width=\textwidth]{logboek}
    \label{fig:logboek}
    \caption{Fragment van het dagelijks logboek met voortgang}
\end{figure}

Na het opstellen van het definitief afstudeerwerkplan wordt er ook een wekelijkse planning gemaakt in de takenlijst van het mailadres van de opdrachtgever. Op deze manier kunnen de begeleiders meekijken hoe de week gevuld wordt, aan welke competenties gewerkt wordt, welke informatie gecommuniceerd gaat worden, belangrijke datums etc.

\newpage
Doordat de urennorm wordt gehanteerd voor de voortgang, betekent dit ook dat er wordt gehouden aan de gestelde mijlpalen die voorgesteld worden in de afstudeerhandleiding. Er zijn echter enkele afwijkingen, de opdracht is namelijk gestart in kalenderweek 36 dus er wordt gestreefd om alle mijlpalen een week eerder af te hebben. Daarnaast zijn er in dit AWP drie deelvragen geformuleerd, de onderlinge tijdsverdeling tussen de deelvragen zal hier dus afwijken. Deze planning is thuis en op de werkplek snel voorhanden zodat snel kan worden gezien of het tijdsschema op de lange termijn nog overeen komt.

\begin{figure}[!ht]
    \centering
    \includegraphics[angle=0,width=\textwidth]{planning}
    \label{fig:planning}
    \caption{Algemene planning afstuderen}
\end{figure}

Communicatie met de afstudeerdocent vindt plaats wanneer tijdens de uitwerking van het product- of procesverslag tegen problemen aan wordt gelopen waar teveel uren verloren gaan door het probleem zelf op te lossen of in overleg met de afstudeerbegeleiders. Het contact met de docent zal dus beperkt zijn, maar uiterst behulpzaam wanneer het nodig is. Contact met de afstudeerdocent gebeurt ook bij afgesproken momenten als het bedrijfsbezoek of voor de reflectie op de functioneringsgesprekken.

Communicatie met de afstudeerbegeleiders gebeurt wekelijks tijdens een vast moment op de vrijdagmiddag. Er wordt dan gereflecteerd op de werkweek en er wordt nadere bedrijfsinformatie gegeven die nuttig zal blijken aan de hand van de gemaakte planning voor de aankomende week. Deze besprekingen zijn tot nu toe tussen anderhalf en twee uur geduurd, naarmate de afstudeerstage vordert zal dit waarschijnlijk minder worden. Elke twee weken wordt een kort reflectiegesprek gehouden waarin mijn functioneren wordt besproken, dit om te voorkomen dat er bij het daadwerkelijke functioneringsgesprek geen verassingen komen. Voor korte vragen waar zelf geen antwoord op gevonden kan worden, wordt de eerste afstudeerbegeleider (J. Molenaar) ingeschakeld. Voor de grote lijnen wordt ook de tweede begeleider betrokken (L. Brouwer).