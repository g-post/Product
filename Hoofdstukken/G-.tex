\chapter{Competentieontwikkeling}
\section{Persoonlijke ontwikkelpunten}
Ten behoeve van het persoonlijk functioneren en ontwikkelen voor de afronding van de bachelor opleiding accountancy, zijn de volgende ontwikkelpunten geformuleerd:
\begin{enumerate}
    \item Er worden stappen gezet om de communicatieve vaardigheden in woord en geschrift te versterken door voor interviews (met zowel de opdrachtgever als andere betrokkenen) goed na te denken over welke informatie benodigd is en hoe deze op een zo'n neutraal mogelijke wijze kan worden geformuleerd. Taalgebruik in rapportage wordt verbeterd door meerdere reviewers te vragen kritisch te kijken naar de rapportage en te doorgronden welke fouten worden gemaakt om verdere fouten te voorkomen. 
    \item Er wordt volgens een vast tijdsplan gestudeerd en gewerkt zodat er een dwangmatige urgentie is om door te blijven werken. Dit wordt gerealiseerd door vaste routines op te bouwen en periodieke mijlpalen te stellen. Periodieke reflectie op de mijlpalen is hier ook belangrijk.
    \item Er wordt beter nagedacht over de toegevoegde waarde richting een potentiële werkgever en deze wordt op een eenduidige wijze geformuleerd. Dit wordt bereikt door te werken aan zo veel mogelijk werkzaamheden die relevant zijn met de accountancy opleiding als achtergrond. Concreet moet worden geformuleerd welke waarde kan worden toegevoegd die een andere afgestudeerde AC-student niet zou kunnen aanbieden.
\end{enumerate}

\newpage
\section{Landelijke beroepscompetenties}
Op de Windesheim community zijn een aantal onderzoekscompetenties geformuleerd om de afstudeerstudent te helpen bij het succesvol afronden van de scriptie. Een aantal hier van hebben betrekking op dit afstudeerwerkplan. Hier wordt verder toegelicht hoe deze competenties aan de orde zijn gekomen bij het opstellen van het AWP. \\
De afgestudeerde bachelor AC-student is in staat om: 
\begin{enumerate}
    \item Een probleemstelling te formuleren
    \item Voor de oplossing van het probleem beschikbare kennis te identificeren
    \item De kwaliteit van de kennis en de theorie die hem ter beschikking staat kritisch te beoordelen
    \item De basis van zijn analyse te modelleren
\end{enumerate}

\noindent
\textit{Ad. 1.} Er is hard en lang over de probleemstelling en aanleiding gediscussieerd met de opdrachtgever en gedeeltelijk met begeleidende docenten. De aanleiding was in de eerste fase van de opdrachtsformulering niet helemaal helder en er moest geschaafd en bijgesteld worden om een goede aanleiding en probleemomschrijving te formuleren. Het lijkt aan de ene hand tegenstrijdig, omdat wel meteen duidelijk was wat de doelstelling was en welke producten opgeleverd moeten worden. Juist door zo hard te denken over de probleemstelling in samenwerking met de betrokken partijen, is de huidige probleemstelling zo hard en onderbouwd. 

\bigskip
\noindent
\textit{Ad 2.} Alvorens het afstudeerwerkplan is opgesteld is er diep en breed gezocht naar enigszins relevante informatie uit een groot aantal soorten bronnen. Andere scripties, toegewijde vakliteratuur, en handreikingen vanuit het opleidingsinstituut waren hier het meest behulpzaam. De term \textit{treasury} (zie paragraaf \ref{def:treasury}) was voor deze verkenning niet compleet duidelijk. Er is voor dit onderzoek veel ingelezen over geldmiddelenbeheer om toch een voldoende geïnformeerde uiteenzetting te geven in de probleemomschrijving. 

\bigskip
\noindent
\textit{Ad. 3.} Er is, met afstemming met de opdrachtgever, een realistisch te behalen eindproduct voor ogen gesteld. De afkadering en de van de grenzen die zijn gesteld zijn bij beide partijen duidelijk en geaccepteerd.

\bigskip
\noindent
\textit{Ad. 4.} Het proces van het formuleren van de hoofdvraag en passende deelvragen was heel tijdrovend en was onderhevig aan veel veranderingen. Het uitwerken van meerdere mindmaps en causale veldmodellen hielpen hier enorm bij. Het zien van onderlinge verbanden was opbouwend en hielp om helder te zien hoe de opdracht opgedeeld zou moeten worden en welke informatie en deelstappen daarvoor nodig zouden moeten zijn.
\citep{competenties}