%Wat zijn de sterke en zwakke punten van de controle-omgeving bij Seafood Connection?

\subsection{Theorie}


Delen:
- Cultuur binnen de organisatie
- Toezicht op de leiding
- Houding ten opzichte van interne controle
- Controleplichtigheid
- Wet- en regelgeving binnen de branche
- Economische omstandigheden binnen de branche en voor de organisatie in bijzonder
- Het al dan niet hebben van een code of conduct
\citet{bivpraktijk}

- Integriteit en ethische waarden
(In hoeverre zijn integriteit en ethische waarden een belangrijk uitgangspunt?)
- Beloningsstructuur
- Voorbeeldgedrag van de leiding
- Opleiding en competenties van de medewerkers van de organisatie
- Kwaliteit van het toezicht dat wordt uitgeoefend op de organisatie door de Raad van Commissarissen
(In hoeverre is de RvC onafhankelijk van het management en oefent deze effectief toezicht uit op de opzet en ontwikkeling van het interne beheersingssysteem?)
- Manier van leidinggeven
- Wijze waarop de organisatie gestructureerd is
(In hoeverre sluit de structuur van de organisatie en de verdeling van bevoegdheden aan op de gestelde doelen?)
- Delegatie van verantwoordelijkheden en bevoegdheden
(In hoeverre worden individuen verantwoordelijk gehouden voor de voorschriften van het interne beheersingssysteem?)
- Beleid op het gebied van personeelszorg 
(In hoeverre is de organisatie in staat om competente medewerkers aan te trekken te onwikkelen en te behouden die de organisatiedoelen kunnen verwezenlijken?)
\citet{bivperspectief}



COSO-principles

\begin{table}[h!]
    \centering
    \caption{De principes voor effectieve interne controle volgens COSO}
    \begin{tabular}{l l}
        \toprule
        \textbf{Component} & \textbf{Principle} \\
        \midrule
        Control environment & 1. Demonstrate commitment to integrity \\
         & and ethical values \\
         & 2. Ensure that board exercises oversight \\
         & responsibility \\
         & 3. Establish structures, reporting lines, \\
         & authorities and responsibilities \\
         & 4. Demonstrate commitment to a competent \\
         & workforce \\
         & 5. Hold people accountable \medskip \\
        Risk assessment & 6. Specify appropriate objectives \\
         & 7. Identify and analyze risks \\
         & 8. Evaluate fraud risks \\
         & 9. Identify and analyze changes that could \\
         & significantly affect internal controls \medskip \\
        Control activaties & 10. Select and develop control activities that \\
         & mitigate risks \\
         & 11. Select and develop technology controls \\
         & 12. Deploy control activities through policies \\
         & and procedures \medskip \\
        Information and communication & 13. Use relevant, quality information to \\
         & support the internal control function \\
         & 14. Communicate internal control information \\
         & internally \\
         & 15. Communicate internal control information \\
         & externally \medskip \\
        Monitoring activities & 16. Perform ongoing or periodic evaluations of \\
         & internal controls (or a combination of the two) \\
         & 17. Communicate internal control deficiencies \\
        \bottomrule
    \end{tabular}
    \label{tab:cosoprincipes}
\end{table}

\subsection{Uitwerking}


\subsection{Conclusie}