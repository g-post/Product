\chapter{Bedrijfsachtergrond}
\section{De onderneming}
\subsection{Omschrijving}
Opgericht in 1995, door huidig CEO J. (Jan) Kaptijn en wijlen C. (Chris) Goos met het innovatieve idee om een geselecteerd assortiment visproducten aan te bieden over de hele wereld zonder zelf ook maar een enkele visverwerkingsband te moeten aanschaffen. Seafood Connection werkt nauw samen met landen over de hele wereld om lokaal visproducten aan te kunnen bieden met een hoge kwaliteitsstandaard. Deze standaard wordt gegarandeerd door contacten van het bedrijf zelf grondige controles uit te laten voeren bij de leveranciers op locatie. Daarnaast wordt het productie- en verwerkingsproces gehouden aan hoge Europese standaarden als IFS, BRC, MSC en HACCP. \citep{sfcwebsite}

In 2013 werd de meerderheid van Seafood Connection Holding gekocht door het Japanse Maruha Nichiro. Dit miljardenbedrijf, opgericht in 1880, is evenals SFC in alle uithoeken van de wereld te vinden. SFC hoopt met deze samenwerking te kunnen profiteren van de kennis en middelen van Maruha Nichiro, die naast de verkoop van visproducten het doel hebben om de hele waardeketen van de visindustrie te domineren. Dit betekent dat Maruha Nichiro een hand heeft in een groot aantal bedrijven dat rijkt van bedrijven in de aquacultuur, distributiecentrums en fabrieksschepen. Het moederbedrijf houdt het Nederlandse Seafood Connection nauwlettend in de gaten, dit doet zij door werknemers van Maruha Nichiro bij Seafood Connection op locatie te laten werken zodat zij periodieke rapportage kunnen versturen naar het hoofdkantoor in Japan. \citep{sfcwebsite,Visserijnieuws}

\subsection{Fase}
Seafood Connection opereert in een markt die deels gedefinieerd wordt door prijsconcurrentie, het leveren van visproducten met kwaliteitskeurmerken is niet een nieuw fenomeen. Het onderscheidend karakter moet vaak ergens anders vandaan komen dan het voldoen aan minimum kwaliteitseisen. 

Het onderscheidend karakter is bij SFC bij een aantal zaken te merken. Ten eerste is er door de jaren heen zoveel kapitaal verworven dat SFC in een bevoorrechte positie is waar zij op grote schaal kan inkopen. Tevens heeft SFC een select aantal leveranciers die nauwkeurig gekozen zijn op hun bereidbaarheid om vis te leveren die aan de normen SFC voldoen. SFC opereert in een verzadigde markt, maar groeit altoos door haar marktaandeel en het productaanbod.

\subsection{De activiteiten in perspectief} \label{beschr:activiteiten}
Om een volledig beeld te krijgen van de ondernemingsactiviteiten is het belangrijk om aandacht te besteden aan de financiële kant van de onderneming. Hier gebeurt namelijk iets opmerkelijks, wanneer bijvoorbeeld de financiële ratio's vergeleken worden met algemeen geaccepteerde normen scoort Seafood Connection, voor een handelsbedrijf, relatief slecht. Om deze ratio’s te beoordelen moeten zij eerst in context worden geplaatst, deze kunnen niet één-op-één vergeleken worden met normen die worden gehanteerd voor een regulier handelsbedrijf. Het is voor een bedrijf als SFC namelijk aantrekkelijk om relatief veel schulden op zich te nemen ten opzichte van haar bezittingen. SFC koopt grootschalig in voor verscheidene producten op een groot aantal markten. Dit betekent dat wanneer een grote hoeveelheid visproducten worden gekocht, het gebruikelijk is dat de levering doorgaans pas na één maand plaatsvindt. Na levering blijft de voorraad doorgaans twee maanden op voorraad voor het verkocht wordt, waarbij betaling van deze order één tot twee maanden na de aflevering van deze goederen plaatsvindt. In feite ontvangt Seafood Connection vaak pas na vier tot zes maanden na inkoop van het product pas haar geld terug. De ratio’s die hiermee verbonden zijn kunnen dus niet ontkoppeld worden aan de inherente eigenschappen van bulkinkoop op een internationale markt. \citep{jaarrapport2017}

\begin{table}[h]
    \centering
    \caption{Kengetallen jaarrekening over 2017 \citep{jaarrapport2017}}
    \begin{tabular}{l r r}
        \toprule
        \textbf{Kengetal} & \textbf{Ultimo 2017} & \textbf{Ultimo 2016} \\
        \midrule
        Rendament TV & 5,6\% & 4,9\% \\
        Solvabiliteit & 24,6\% & 20,8\% \\
        Liquiditeit & 0,55 & 0,53 \\
        \bottomrule
    \end{tabular}
    \label{tab:kengetallen}
\end{table}

Uit het jaarrapport over 2017 worden de financiële ratio’s uit tabel \ref{tab:kengetallen} ontleend. Deze cijfers gelden voor de peildatum van respectievelijk 31 december 2017 en 2016.

Het rendement op het totale vermogen is berekend als ((nettowinst + betaalde rente) / balanstotaal), de solvabiliteit als (eigen vermogen / totaal vermogen), en de liquiditeit als ((vlottende bezittingen + liquide middelen) / kortlopende schulden). \citep{jaarrapport2017}

\newpage
    \section{De organisatie}
Het is belangrijk om te verkenning hoe de organisatie is opgebouwd, niet alleen omdat dit een opdracht is die kijkt naar de administratieve \textit{organisatie}, maar ook omdat de organisatie van Seafood Connection veranderd is sinds de laatste keer dat er kritisch is gekeken naar de AOIB. Sinds die tijd is er haast een verdubbeling geweest van het aantal medewerkers van Seafood Connection B.V., de omzet is verdubbeld, en de indeling van afdelingen is veranderd door deze opnieuw in te richten zodat zij onder andere beter kunnen focussen op deelmarkten, (nieuwe) producten, en geografische gebieden. Om dit alles overzichtelijk te maken voor de probleemomgeving wordt in dit hoofdstuk de organisatie van Seafood Connection B.V. uitgewerkt. 

\subsection{Typering}
Seafood Connection is een handelsbedrijf in diverse diepvries visproducten met daarnaast in beperkte mate doorstroom van eigen goederen met een eenvoudig, technisch omzettingsproces \citep{aoibsfc}. Voor de verschillende bedrijfsactiviteiten fungeert Seafood Connection als:

\begin{enumerate}
    \item Handelsbedrijf dat hoofdzakelijk aan andere bedrijven levert
    \item Productiebedrijf met homogene massaproductie
    \item In zeer beperkte mate dienstverlening aan derden (door commissie op verkopen aan derden)
\end{enumerate}

Volgens de modellen van Starreveld is SFC een \textit{Handelsbedrijf op rekening}. Belangrijke aanknopingspunten voor de interne controle zijn de geautoriseerde brutomarge en prijsprocedure; de harde functiescheiding tussen de afdelingen: inkoop, opslag, verkoop en administratie; én de inventarisatie als het sluitstuk van de geld-goederenbeweging. In de grotendeels zelf-gerapporteerde quick scan geeft SFC zelf aan dat procedures verankerd in het systeem liggen, maar tegelijkertijd flexibel zijn voor optimalisatie; er is een sterke functiescheiding tussen inkoop, opslag en administratie. Echter geeft zij zelf aan dat er sprake is van een open magazijn, dat niet in eigen beheer is, en door derden alleen beheerd wordt door de magazijnmeester, terwijl het magazijn door verschillende mensen wordt geïnventariseerd, en ook er is sóms onafhankelijke controle van de verkoopprijzen. \citep{bivperspectief,quickscan}

Een ander belangrijk detail voor de Administratieve Organisatie is dat het bedrijf in sterke mate gebruik maakt van automatisering. Zo is er een bevoegdhedenmatrix waarin beschikkende medewerkers geautoriseerd of geblokkeerd zijn om bepaalde handelingen uit te voeren. Hier valt de opmerking te plaatsen dat er voor het afschermen van FX-risico's geen limieten zijn voor de hoeveelheid in te kopen valuta. Dit is opzettelijk gekozen zodat de inkopers snel kunnen handelen bij veranderende koersen en niet een bureaucratisch proces hoeven te doorlopen om grote bedragen in te kopen. In de quick scan geeft de organisatie weer dat (bijna) altijd gebruik wordt gemaakt van automatisering. \citep{quickscan}

\subsection{Inrichting}
Seafood Connection is vijftig gemotiveerde werknemers sterk. Het kantoor op Urk vervult alle bedrijfsfuncties grotendeels op één locatie; er zijn enkele medewerkers werkzaam bij een klein, lokaal productieproces, namelijk bij de zagerij van Coldstore Urk waar een groot deel van de voorraad van SFC zich bevindt. De organisatie heeft vier niveaus: het managementteam (MT) dat de strategie formuleert, de unitmanagers (UMO) die hun afdelingen aansturen, managers die het aanspreekpunt zijn voor hun deelprocessen in de verschillende afdelingen, én assistants die de afdelingen ondersteunen met verschillende werkzaamheden. SFC is een lijn-staforganisatie met een gedeelde P-, G-, en M-indeling opgedeeld in de afdelingen: inkoop, opslag, verkoop, finance, ICT and production, HRM, compliance, en marketing verdeeld over vier segmenten: wholesale, retail, industry, en group companies (geografisch beheer). \citep{quickscan}
\section{De branche}
Kenmerken voor succes in de branche zijn ogenschijnlijk het hebben van een affiniteit voor duurzaamheid, naleving van kwaliteitsstandaarden en goed toegankelijk zijn zodat visproducten in elk jaargetijde geleverd kunnen worden. In mindere mate is er ook sprake van prijsconcurrentie, omdat alle visverwerkers uit dezelfde vaargebieden de vis vangen. 

Seafood Connection is niet de enige aanbieder van vis. SFC opereert op Urk, één van de grootste centrums voor de visverwerking in Nederland, de CFO van Seafood Connection zegt hier zelf over dat het een gezonde hoeveelheid concurrentie voor de kiezen heeft. Het is dan ook niet verrassend dat de onderneming als grootste succescriteria heeft gekozen om te winnen in de markt, dit plaatst zij boven criteria als het beschikken over een zo uniek en nieuw mogelijk productassortiment of efficiënte bedrijfsvoering. 

De missie van Seafood Connection is om het volgende te anticiperen: klantbelangen, de behoefte aan duurzame visproducten, en nieuwe trends. De onderneming wil dit realiseren door nieuwe partners te verkrijgen in cruciale markten in Europa, Amerika en Azië. Tevens worden nieuwe keteninnovaties verkregen door fusies en overnames in de visindustrie. \citep{sfcwebsite}