\chapter{Onderzoeksontwerp}

\section{Theoretische ondersteuning}
De uitwerking van de deelvragen wordt gedaan aan de hand van het COSO Internal Control Framework \citep{COSOsummery}. De daadwerkelijke invulling hiervan wordt aan de hand van de uiteenzetting in de literatuur van \citet{bivpraktijk} en die van \citet{bivperspectief} verder verdiepend uitgewerkt. 

Voor de uitwerking van het COSO-model wordt ook gesteund op de theorie van Starreveld waarin per organisatietypologie onder andere wordt uitgewerkt welke preventieve en repressieve maatregelen verwacht worden \citep{jans,financiering,buunk}. Door deze theorie slim te benutten en verschillende typologieën te combineren zal het mogelijk worden om een relevant theoretisch kader voor SFC te maken. 

Samenvattend wordt het COSO-model uitgewerkt om de organisatie als geheel te kunnen omschrijven en om de centrale hoofdvraag te kunnen beantwoorden. De uitwerkingen van de bestuurlijke informatieverzorging van \citet{bivperspectief} en van \citet{bivpraktijk} worden gebruikt voor de interpretatie van het COSO-model. Voor de specifieken van elke laag in het COSO-model en als onderbouwing en uitwerking van de verschillende afdelingen en processen worden de boeken gebruikt omtrent procesbeheersing en de specifieke relevante onderwerpen. \citep{internebeheersing,jans,financiering,buunk}

\section{Het vakgebied}
Deze onderzoeksopdracht valt onder de noemer 'betrouwbaarheid breed'. Deze is gericht op het inrichten van het interne betrouwbaarheidssysteem in het algemeen. Deze opdracht wordt voornamelijk toegepast bij organisaties die sterk aan het groeien zijn en die graag willen weten of de bestaande interne controlemaatregelen nog voldoende zijn. Een veel voorkomende opdracht hierbij is dat het bestaande AOIB handboek verouderd is. \citep{bivpraktijk}


\newpage
\section{Voorwaardelijke onderzoeksvragen}
\begin{enumerate}
    \item Controle-omgeving
        \begin{itemize}
            \item Hoe kan de cultuur van Seafood Connection worden getypeerd?
            \item Wat is de houding van de directie ten opzichte van interne controle?
            \item Wat is de kwaliteit van toezicht op de leiding?
            \item Hoe zijn de economische omstandigheden binnen de branche en de regio? Welke technologische ontwikkelingen hebben een sterke invloed op de branche?
            \item Heeft Seafood Connection een personeelsreglement of code of conduct? Wordt deze genoeg bekend gemaakt bij medewerkers en wordt deze periodiek geëvalueerd?
            \item Welke wetten en regels zijn specifiek voor de branche van Seafood Connection?
            \item Wat is de wijze van belonen van medewerkers?
            \item Wat is de kennis en kunde van de medewerkers ten aanzien van de gestelde strategische doelen en geformuleerde missie?
            \item Is SFC controleplichtig? Speelt dit een rol in organisatorische vormgeving en verslaggeving (mede met het oog op compliance)? Speelt de grootte van de organisatie een rol?
            \item Wie zijn de kernspelers in de geld-goederenbeweging?
        \end{itemize}
    \item Risicoanalyse
        \begin{itemize}
            \item Is er een zekere mate van risicobewustzijn binnen de organisatie?
            \item Welke betrouwbaarheidsrisico's zijn te onderkennen gezien de typologie van Seafood Connection en de waardekringloop? Worden deze risico's herkend in de werkelijkheid?
            \item Welke betrouwbaarheidsrisico's zijn te onderkennen door de jaarrekening te onderzoeken?
            \item Welke betrouwbaarheidsrisico's zijn te onderkennen door de aanwezige processen te onderzoeken? Is er een zeker verband tussen de afdelingen die extra betrouwbaarheidsrisico's met zich meebrengen (bijvoorbeeld de manier waarop wordt omgegaan met het geldmiddelenbeheer)?
        \end{itemize}
    \item Preventieve en repressieve maatregelen van interne controle
        \begin{itemize}
            \item Welke preventieve maatregelen zijn getroffen? (zie tabel \ref{tab:icmaatregelen})
            \item Welke repressieve maatregelen zijn getroffen? (zie tabel \ref{tab:icmaatregelen})
        \end{itemize}
    \item Informatie en communicatie
        \begin{itemize}
            \item Welke informatie is voorhanden aan directie over belangrijke betrouwbaarheids- en bedrijfsrisico's?
            \item Wat is de kwaliteit van de communicatie binnen de organisatie?
            \item Wat is de kwaliteit van de communicatiesystemen binnen de organisatie?
            \item Wordt er voldaan aan de compliance voor externe verslaggeving? 
            %Volgen van NL-GAAP in de externe jaarrekening met betrekking tot verwerking in te kopen / ingekochte vreemde valuta. Wordt nu verwerkt als \textit{niet uit de balans blijkende verplichtingen}, maar er hangt natuurlijk ook een niet uit de balans blijkend recht aan
        \end{itemize}
    \item Monitoring (of: borging)
        \begin{itemize}
            \item Worden de voorschriften en regels met betrekking tot interne betrouwbaarheid nageleefd?
            \item Is het interne betrouwbaarheidssysteem nog steeds actueel en relevant voor de organisatie?
        \end{itemize}
\end{enumerate}


\begin{table}[!h]
    \centering
    \caption{Het stelsel van maatregelen van interne controle \citep{bivpraktijk}}
    \begin{tabular}{l l}
        \toprule
        \textbf{Preventieve maatregelen van IC} & \textbf{Repressieve maatregelen van IC} \\
        \midrule
        Begrotingen, normen, en tarieven & Cijferbeoordeling \\
        Functiescheiding & Verbandscontroles \\
        Procedures en richtlijnen & Detailcontroles \\
        Beveiliging van waarden & Waarneming ter plaatse \\
        Preventieve IT-controls & Repressieve IT-controls \\
        Inrichting van de administratie \\
        \bottomrule
    \end{tabular}
    \label{tab:icmaatregelen}
\end{table}