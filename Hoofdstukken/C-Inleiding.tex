\chapter*{Inleiding} %Ongenummerd hoofdstuk
\addcontentsline{toc}{chapter}{Inleiding} %Ongenummerde chapters komen niet in de ToC, met deze code wel
Deze afstudeeropdracht geeft antwoord op de vraag hoe de interne controle is ingericht bij Seafood Connection te Urk en hoe deze gebruikt kan worden om op een verantwoorde en navolgbare wijze om te gaan met de beschikbare geldmiddelen. In samenwerking met de finance afdeling en het management van Seafood Connection B.V. (hierna als afkorting: SFC) is kritisch gekeken naar de inrichting van de bestaande AOIB. 

Om de lezer een goed beeld te geven van de opdrachtomgeving is eerst beknopt de bedrijfsachtergrond beschreven. 
Deze dient als context voor hoofdstuk \ref{hoofdstuk:probleemanalyse} waarin de centrale probleemstelling is beschreven. Vervolgens is de methodiek voor het onderzoek uitgewerkt in hoofdstuk \ref{hoofdstuk:methodiek}. De resultaten die gevonden zijn aan de hand van het onderzoek komen aan de orde in hoofdstuk \ref{hoofdstuk:resultaten}. De gevormde conclusie en verdere bespreking van deze resultaten is te vinden in hoofdstuk \ref{hoofdstuk:conclusie} waarin ook de centrale hoofdvraag beantwoord is. Als slot vindt u de bronnenlijst, en de bijlagen die verdiepend bedoeld zijn voor deze rapportage. 