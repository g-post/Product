\chapter*{Inleiding}
\addcontentsline{toc}{chapter}{Inleiding} %Ongenummerde chapters komen niet in de ToC, met deze code wel
Om duidelijk de uit te voeren opdracht te definiëren en te verkennen wordt in dit verslag met een brede blik gekeken de probleemformulering, -omgeving, en er wordt onderzocht wat er concreet gaat spelen bij dit onderzoek. In samenwerking met de finance afdeling van Seafood Connection (hierna als afkorting: SFC) wordt georiënteerd op de opdrachtformulering en wordt ook de opdrachtomgeving bekeken aan de hand van een uitgebreide bedrijfsverkenning. 

Om de lezer een goed beeld te geven van de opdrachtomgeving wordt eerst uitgebreid, maar doelgericht, de bedrijfsachtergrond beschreven. Vervolgens wordt de eerste stap van de onderzoekscyclus uitgewerkt volgens het model van Nel Verhoeven waar de probleemomschrijving breed en diep wordt uitgewerkt. In hoofdstuk drie wordt dit probleem omgezet in een concreet ontwerp waarin concreet wordt nagedacht over de invulling van de afstudeeropdracht- en periode. De competentieontwikkeling in hoofdstuk vier beschrijft de manier waarop de afstudeerstagiair gaat werken aan de competentie- en ontwikkelingspunten. Als slot vindt u de bronnenlijst, de bijlagen die verdiepend zijn voor dit plan, en de tijdsplanning voor de afstudeerperiode. 