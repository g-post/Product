%Deze Tex is voor woorden die achterin het rapport duidelijker worden uitgelegd

\makeglossaries

\newglossaryentry{treasuryletter}
{
    name=treasury letter,
    description={Een door het management opgesteld rapport waarin de bevoegdheden, verantwoordelijkheden, toezichtmaatregelen, de sturing, en beheersing geformuleerd worden ten aanzien van financiële vermogenswaarden, financiële geldstromen, financiële posities en de hieraan verbonden risico’s}
}

\newglossaryentry{treasury}
{
    name=treasury,
    description={Het beleid en het beheersen van de geldmiddelen (treasures)}
}

\newglossaryentry{ic}
{
    name=interne controle,
    description={Controle door of namens de leiding en toezichthoudende organen op de oordeelsvorming en actviteiten van anderen}
}

\newglossaryentry{lsorganisatie}
{
    name=lijn-staforganisatie,
    description={Een lijnorganisatie is een organisatie waarbij boven elke werknemer een manager of meerdere staat en waarin de taken opgedeeld zijn in logisch bij elkaar horende afdelingen. Hier worden stafafdelingen aan toegevoegd die uitsluitend adviesbevoegd zijn ten behoeve van leidinggevende functionarissen of directie}
}

\newglossaryentry{geldgoederen}
{
    name=geld-goederenbeweging,
    description={De primaire waardekringloop van een onderneming; het proces van het opofferen en verkrijgen van waarden. Zie figuur \ref{fig:primairproces}}
}

\newglossaryentry{fx}
{
    name=FX,
    description={Valuta-, rente- en koersrisico. Zie aldaar}
}

\newglossaryentry{biv}
{
    name=bestuurlijke informatieverzorging,
    description={Alle activiteiten met betrekking tot het systematisch verzamelen, vastleggen, en verwerken van gegevens gericht op het verstrekken van informatie ten behoeve van het nemen van beslissingen, het dagelijks functioneren, het beheersen van de organisatie, en het afleggen van verantwoording}
}

\newglossaryentry{intbet}
{
    name=interne betrouwbaarheid,
    description={De mate van zekerheid rondom: de betrouwbaarheid van opgestelde financiële verantwoordingen, dat medewerkers zicht houden aan interne regels en dat de organisatie voldoet aan wet- en regelgeving, én dat waarden niet ongeoorloofd de organisatie verlaten}
}

\newglossaryentry{intbeh}
{
    name=interne beheersing,
    description={In staat zijn om activiteiten beheerst te laten verlopen. Van een beheerst proces is sprake bij een meetbare norm,het meten van de wekelijke resultaten ten opzichte van deze norm, en het bijsturen van het proces na het vergelijken van de werkelijkheid met de norm}
}

\newglossaryentry{valuta}
{
    name=valutarisico,
    description={Risico dat de wisselkoers van vreemde valuta verandert zodat een vordering of schuld, in vreemde valuta, nadelig is gewijzigd}
}

\newglossaryentry{rente}
{
    name=renterisico,
    description={Risico dat er ongewenste veranderingen optreden van de (financiële) resultaten door rentewijzigingen}
}

\newglossaryentry{koers}
{
    name=koersrisico,
    description={Risico dat de koers van een financieel instrument een ongewenste beweging maakt, waardoor er verlies wordt gemaakt op deze instrumenten}
}

\newglossaryentry{typologie}
{
    name=betrouwbaarheidstypologie,
    description={Typologie die per type bedrijf aangeeft welke interne betrouwbaarheidsmaatregelen bij dat bedrijf verwacht kunnen worden. Synoniem voor Starreveld-typologie}
}

%