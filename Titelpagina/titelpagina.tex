%Dit is een standalone TeX. Deze wordt gecompiled als main.tex een andere preamble heeft als dit bestand, bijvoorbeeld oneside op titelpagina, twoside in main.tex

\documentclass[10pt,a4paper,oneside]{report}
\usepackage[utf8]{inputenc} %Codering
\usepackage[dutch]{babel} %Taalinstelling
\usepackage{biblatex} %Om een of andere reden wil deze TeX niet compilen zonder deze package, deze is niet nodig
\usepackage{csquotes} %Deze is ook nodig om errors tegen te gaan, hoewel hij zelf niet nodig is

\usepackage{graphicx}
\graphicspath{ {../Images/} }

\usepackage[linktoc=all]{hyperref}
\hypersetup{colorlinks=false} %Links naar websites, bronnen, inhoudsopgave, pagina's etc.

\usepackage[pages=some]{background}

\usepackage[a4paper, margin=2.5cm]{geometry}

\usepackage{titling} %Titels voor op de titelpagina
\def\titel{Het interne betrouwbaarheidssysteem en haar invloeden op het geldmiddelenbeheer}
\def\ondertitel{Bachelor afstudeeropdracht over de interne controle van \\
Seafood Connection B.V. en de invloed hier van op het geldmiddelenbeheer}

\def\auteur{Gerrit Post}
\def\studentklas{S1071236 --- AC4V}
\def\mailstudent{S1071236@student.windesheim.nl}
\def\telstudent{+31 642 678 172}

\def\school{Hogeschool Windesheim te Zwolle}
\def\domein{Accountancy, BMR}
\def\mailschool{a.vanden.brandhof@windesheim.nl}

\def\organisatie{Seafood Connection B.V. te Urk}
\def\mailorganisatie{info@seafoodconnection.nl}
\def\telorganisatie{+31 527 687 066}

\def\docent{drs. A. Dannenberg RA}
\def\begeleidereen{dhr. J.J. Molenaar, MSc.}
\def\begeleidertwee{dhr. L. Brouwer}

\def\voecode{ACvAFST.1819}
\def\perjaar{najaar 2018}
\def\datum{7 januari 2019}
\def\plaats{Urk}

\def\vertspace{0.5cm}

\title{\titel}
\author{\auteur}
\date{\datum}
%EIND VAN PREAMBLE


\begin{document}
%\backgroundsetup{contents=Concept,opacity=0.25}
%\BgThispage
\begin{titlepage}
    \vspace*{-0.9cm}
    \hfill
    \includegraphics[width=0.35\textwidth]{windesheim} \\
    
    \begin{center}
    \vspace*{3cm}
    {\huge\thetitle}
    
    \vspace*{0.4cm}
    \textnormal{\ondertitel}
    \end{center}
    
    \raggedleft
    \vfill
    \textbf{Door} \\
    \theauthor \\
    \studentklas \\
    \mailstudent \\
%   \telstudent \\
    
    \vspace{\vertspace}
    \textbf{Onderwijsinstituut} \\
    \school \\
    \domein \\
%   \mailschool \\

    \vspace{\vertspace}
    \textbf{Traject} \\
    \voecode \\
    \perjaar \\
    
    \vspace{\vertspace}
    \textbf{Organisatie} \\
    \organisatie \\
    \mailorganisatie \\
%   \telorganisatie \\

    \vspace{\vertspace}
    \textbf{Begeleiding} \\
    Eerste afstudeerbegeleider: \begeleidereen \\
    Tweede afstudeerbegeleider: \begeleidertwee \\
    Afstudeerdocent: \docent \\
\end{titlepage}
%Titelpagina

\end{document}